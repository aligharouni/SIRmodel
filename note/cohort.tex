\documentclass[12pt]{article}
\usepackage{natbib}
\usepackage{hyperref}
\usepackage{grffile}
\usepackage{graphicx}
\usepackage{subcaption}
\usepackage{amssymb,amsmath,amsthm}
\usepackage{xcolor}
\usepackage{xspace}
\usepackage[nameinlink,capitalize]{cleveref}
\usepackage{cleveref}
\usepackage[margin=1in]{geometry}
\usepackage{lineno}\renewcommand\thelinenumber{\color{gray}\arabic{linenumber}}
\usepackage{pdflscape}
\usepackage{enumerate}

\newcommand{\todo}[1]{\comment{red}{TODO}{#1}}

\usepackage{xspace}

\newcommand{\fref}[1]{Fig.~\ref{#1}}
\newcommand{\Rlogo}{R\xspace}
\newcommand{\percap}{\emph{per capita}\xspace}
\newcommand{\Rnot}{\ensuremath{\mathcal{R}_0}}
\newcommand{\covid}{COVID-19\xspace}
\newcommand{\pro}[1][]{\ensuremath{\frac{\partial #1}{\partial \rho}}}
\newcommand\pder[2][]{\ensuremath{\frac{\partial#1}{\partial#2}}} %\pder[x]{y}

\newcommand{\comment}{\showcomment}
\newcommand{\showcomment}[3]{\textcolor{#1}{\textbf{[#2: }\textsl{#3}\textbf{]}}}
\newcommand{\nocomment}[3]{}

\theoremstyle{definition} % amsthm only
\newtheorem{proposition}{Proposition}
\newtheorem{theorem}{Theorem}

\bibliographystyle{apalike}

\title{The analogy of the cohort-equations paradigm to the compartmental epidemic models}

\begin{document}
\maketitle
\linenumbers

\section{Vision}
\begin{enumerate}
\item 2 frameworks of \cite{van2002reproduction}, I call it compartment framework, and \cite{champredon2018equivalence}, I call it cohort framework, can be tied together and a mechanistic approach to go from one framework to another can be constructed.

\item 
\end{enumerate}

\section{Introduction}
Notation; we use $I'$ for the cohort-framework of $dI/d\tau$ and $\dot I$ for the compartment-framework of $dI/dt$. 

A simple example of this transition is the cohort analogy of a simple SIR compartmental model where $\dot I=\beta S I/N-\gamma I$. The cohort-framework is via the cohort Eq. with the following steps
step 1: write the cohort Eq. $I'=-\gamma I$ with the initial condition $I(0)=1$, 
step 2: finding the intrinsic infectiousness kernel by integrating the cohort Eq. and solving for its time evolution, thus
$k(\tau)=\beta \exp(-\gamma \tau)$.
step 3: $R$ will be the integration of the kernel. That is $R=\beta/\gamma$. 
Note that in the comartment framework, steps 2 and 3 are combined.




In Watmough 2002's work matrix form of the I compartment $I'=(-V)I$, where I is n-by-1 and $V$ is n-by-1.

In the context of our SIR model with testing, $I^T=[I_u,I_n,I_p,I_c]$ where $T$ is the transposed operator.
And the solution is given by $I(\tau) = \exp(-V\tau) I(0)$, where $\tau$ is in the infection-time scale.
$K(\tau)=\Rnot g(\tau)= F I(\tau)$,
where $F$ is n-by-n matrix of new infections, $I(0)=[1,0,0,0]$.

$\exp(-V\tau)$ is the probability of being infected and stay infectious at time $\tau$.

$\exp(-V\tau)=\sum_{i=0}^\infty \tau^n (-V)^n/n!$.





\bibliography{../SIRlibrary}
\end{document}