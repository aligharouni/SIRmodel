\documentclass[12pt]{article}
\usepackage{natbib}
\usepackage{hyperref}
\usepackage{graphicx}
\usepackage{subcaption}
\usepackage{amssymb,amsmath,amsthm}
\usepackage{xcolor}
\usepackage{xspace}
\usepackage[nameinlink,capitalize]{cleveref}
\usepackage{cleveref}
\usepackage{geometry}
\usepackage{pdflscape}

\newcommand{\comment}{\showcomment}
%% \newcommand{\comment}{\nocomment}

\newcommand{\showcomment}[3]{\textcolor{#1}{\textbf{[#2: }\textsl{#3}\textbf{]}}}
\newcommand{\nocomment}[3]{}

\newcommand{\fady}[1]{\comment{cyan}{Fady}{#1}}
\newcommand{\ali}[1]{\comment{magenta}{Ali}{#1}}
\newcommand{\jd}[1]{\comment{blue}{JD}{#1}}
\newcommand{\djde}[1]{\comment{red}{DJDE}{#1}}
\newcommand{\bmb}[1]{\comment{red}{BMB}{#1}}
\newcommand{\todo}[1]{\comment{red}{TODO}{#1}}

\newcommand{\Rnum}{\mathcal{R}_0}
\theoremstyle{definition} % amsthm only
\newtheorem{prop}{Proposition}
\newtheorem{note}{note}
\newtheorem{theorem}{Theorem}

\bibliographystyle{apalike}

\title{Notes on modeling targeted testing }

\begin{document}
\maketitle

% %%%%%%%
\section{Review}

\cite{scarabel2021renewal} say: 
Modelling contact tracing is challenging: a satisfactory mathematical description should account at least for concurrent screening/diagnosis programs that can initiate contact tracing, and the contacts between individuals that occurred prior to that moment. For these reasons,
many mathematical models have been formulated as stochastic branching processes [6, 7, 8], or agent-based models [9, 10, 11, 12, 13], which allow to keep
track of the epidemiological status of each individual in the population together
with all their infectious contacts.

The few deterministic models available in the literature [6, 14, 15, 16] are typically obtained as approximations of a corresponding stochastic model under simplifying
assumptions.

% %%%%%%%
\bibliography{../SIRlibrary}
\end{document}
