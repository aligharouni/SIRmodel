\documentclass{article}\usepackage[]{graphicx}\usepackage[]{color}
% maxwidth is the original width if it is less than linewidth
% otherwise use linewidth (to make sure the graphics do not exceed the margin)
\makeatletter
\def\maxwidth{ %
  \ifdim\Gin@nat@width>\linewidth
    \linewidth
  \else
    \Gin@nat@width
  \fi
}
\makeatother

\definecolor{fgcolor}{rgb}{0.345, 0.345, 0.345}
\newcommand{\hlnum}[1]{\textcolor[rgb]{0.686,0.059,0.569}{#1}}%
\newcommand{\hlstr}[1]{\textcolor[rgb]{0.192,0.494,0.8}{#1}}%
\newcommand{\hlcom}[1]{\textcolor[rgb]{0.678,0.584,0.686}{\textit{#1}}}%
\newcommand{\hlopt}[1]{\textcolor[rgb]{0,0,0}{#1}}%
\newcommand{\hlstd}[1]{\textcolor[rgb]{0.345,0.345,0.345}{#1}}%
\newcommand{\hlkwa}[1]{\textcolor[rgb]{0.161,0.373,0.58}{\textbf{#1}}}%
\newcommand{\hlkwb}[1]{\textcolor[rgb]{0.69,0.353,0.396}{#1}}%
\newcommand{\hlkwc}[1]{\textcolor[rgb]{0.333,0.667,0.333}{#1}}%
\newcommand{\hlkwd}[1]{\textcolor[rgb]{0.737,0.353,0.396}{\textbf{#1}}}%
\let\hlipl\hlkwb

\usepackage{framed}
\makeatletter
\newenvironment{kframe}{%
 \def\at@end@of@kframe{}%
 \ifinner\ifhmode%
  \def\at@end@of@kframe{\end{minipage}}%
  \begin{minipage}{\columnwidth}%
 \fi\fi%
 \def\FrameCommand##1{\hskip\@totalleftmargin \hskip-\fboxsep
 \colorbox{shadecolor}{##1}\hskip-\fboxsep
     % There is no \\@totalrightmargin, so:
     \hskip-\linewidth \hskip-\@totalleftmargin \hskip\columnwidth}%
 \MakeFramed {\advance\hsize-\width
   \@totalleftmargin\z@ \linewidth\hsize
   \@setminipage}}%
 {\par\unskip\endMakeFramed%
 \at@end@of@kframe}
\makeatother

\definecolor{shadecolor}{rgb}{.97, .97, .97}
\definecolor{messagecolor}{rgb}{0, 0, 0}
\definecolor{warningcolor}{rgb}{1, 0, 1}
\definecolor{errorcolor}{rgb}{1, 0, 0}
\newenvironment{knitrout}{}{} % an empty environment to be redefined in TeX

\usepackage{alltt}
\usepackage{natbib}
\usepackage{hyperref}
\bibliographystyle{apalike}

\title{SIR Model with Testing and Isolation Mechanisms}
\IfFileExists{upquote.sty}{\usepackage{upquote}}{}
\begin{document}
\maketitle

% %%%%%%%
\section{Method}

% %%%%%%%
\section{Analysis}
    

\section{Literature Review}

% %%%%%%%
\subsection{Explicit models of TTI (trace/test/isolate) based on network or agent-based models}

\citep{endo2020implication} [Ali: It seems to me that this is just a statistical model to estimate the parent-offspring of an infected index, not sure if it fits into agent-based group!] Used simulation on a branching process model to assess the forward and backward contact tracing efficiency. Assuming a negative-binomial branching process with a mean R, reproduction number, and overdispersion parameter k, the mean total number of generation G3 and averted G3 are estimated. The effectiveness of TTI is defined as the ratio of averted to the mean.

\citep{jenness2020modeling} developed a network-based transmission model for SARS-CoV-2 on the Diamond Princess outbreak to characterize transmission dynamics and to estimate the epidemiological impact of outbreak control and prevention measures. 

\citep{elbanna2020entry} [seems similar to MacPan model!]

\citep{de2020influenza} Was discussed in the Math 747 
SEIR Asymptomatic and symptomatic $I_1, I_2$. Used linear chain trick 
Stringency index as a control force lowering $\beta$.

\citep{rice2020effect} Effect of school closures on mortality. Reproduce Report 9 results by spatial agent based CovidSim. 
% %%%%%%%
\subsection{Models of repeated random testing of isolated populations}
\cite{bergstrom2020frequency}
(1) Model, assumptions: They developed a function, namely expected expoisour $E(C,\tau)$, to approximate trade-offs between the frequency of testing, n, the sensitivity of testing, q, and the delay between
testing and results, d. This function is explicitly derived and was connected the effective reproduction number $R=R_0 S$, where $S$ is the proportion of population susciptable.
assumption that transmission rates are a step function: individuals who
have COVID go from non-infectious to fully infectious instantaneously,
and remain fully infectious until they are no longer able to transmit disease. Test sensitivity takes the same form over the course of infection.
More sophisticated models could allow varying infectiousness and varying
sensitivity over time, as in 
\citep{larremore2020test}.

\citep{lopman2020model} Used a Deterministic SEIR model, incorporated TTI, applicable to a university setting. They assumed a fairly high reproductive number that is not reduced through social
distancing measures. They found that community-introduction of SARS-CoV-2 infection onto campus can be
relatively controlled with effective testing, isolation, contract tracing and quarantine.

\citep{tuite2020mathematical} used an age-structured compartmental model of COVID-19 transmission in the population of Ontario, Canada. We compared a base case with limited testing, isolation and quarantine to different scenarios. 




% %%%%%%%
\subsection{Other maybe-related works}
\citep{arino2020simple} developed a SLIAR compartmental model to study the spread of an epidemic, specifically COVID-19, in a population. The model incorporates an Erlang distribution of times of sojourn inincubating, symptomatically and asymptomatically infectious compartments. Basic reproduction number is derived. Also, sensitivity analysis with respect to the underlying parameters for the following two outputs was carried out; (i) the number of observable cases during the course of the epidemic and at the peak, and (ii) the timing of the peak of the outbreak. Sensitivity analysis is performed using the R package multisensi.

\citep{ruszkiewicz2020diagnosis} novel with-in-a-minute breath testing with 80\% accuracy. 
% %%%%%%%
\bibliography{../SIRlibrary}
\end{document}
