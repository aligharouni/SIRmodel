\documentclass[12pt,letterpaper]{letter}

%% layout
\usepackage[margin=1.25in]{geometry}
\usepackage{multicol}

%% colours
\usepackage{xcolor}
\definecolor{linkcolor}{HTML}{0f87ff} % green
\definecolor{irena}{HTML}{c9657d} % blush
\definecolor{de}{HTML}{5681c4} % blue

\usepackage[colorlinks=true, allcolors=linkcolor]{hyperref}

%% date format
\usepackage{datetime}
\newdateformat{mydate}{\twodigit{\THEDAY}{ }\shortmonthname[\THEMONTH] \THEYEAR}

%% comment macros
\newcommand{\ali}[1]{\textcolor{blush}{$\langle$\textbf{AG:} #1$\rangle$}}

%% journal name macro
\newcommand{\journalname}{\emph{BMB}}
% Bulletin of Mathematical Biology
%% data macros
% \input{data_macros.tex}

\setlength{\parskip}{0em}

\begin{document}

\begin{multicols}{2}
\footnotesize
\begin{flushleft}

Dr. Alan Hastings and Dr. Reinhard Laubenbacher

Editor, \journalname{}

\vfill

{\normalsize \mydate
\today}
\end{flushleft}

\columnbreak

\begin{flushright}
Ali Gharouni

Postdoctoral Fellow 

Department of Mathematics and Statistics,

McMaster University,

Hamilton, ON, Canada, L8S 4K1

\href{mailto:agharoun@uottawa.ca}{agharoun@uottawa.ca}
\end{flushright}

\end{multicols}

\setlength{\parskip}{1em}
\thispagestyle{empty}

\vspace{-1em}

Dear Drs.\ Hastings and Laubenbacher,

We are pleased to submit an original research article entitled
``Testing and Isolation Efficacy: Insights from a Simple Epidemic
Model'' for your consideration for publication in \journalname{}.  Our
study provides new insights into the evolving COVID-19 pandemic, and
more generally to epidemic dynamics in the context of testing.

We have developed a compartmental model to understand the effect of testing processes on epidemic dynamics. We studied the effects of testing intensity, rate of test return, and isolation efficacy, on epidemic dynamics when different levels of testing focus are in place. Our analysis shows that under some circumstances both increased testing intensity and faster test reporting can reduce the effectiveness of control, and allows us to explore the conditions under which this occurs. Conversely, we find that focusing testing on infected individuals always acts to increase effectiveness of control. Our work provides insights for epidemiologists, mathematical modelers and public health managers in the context of the interaction between testing processes and epidemic processes.

All authors have approved the manuscript for submission and declare no competing interests. A preprint of this work has been posted to arXiv (\url{https://arxiv.org/abs/2107.08259}). The manuscript has not been published or submitted for publication elsewhere.

Thank you very much for your consideration,

\begin{multicols}{2}
\begin{flushleft}
\footnotesize

Ali Gharouni, PhD
\setlength{\parskip}{0em}

Postdoctoral Fellow, Department of Mathematics and Statistics, McMaster University

\vspace{1em}

Fady M. Abdelmalek, B.Sc.

Department of Mathematics and Statistics, McMaster University

\vspace{1em}

David J.\,D.\ Earn, PhD

Professor, Department of Mathematics \& Statistics, McMaster University

Member, Michael G. DeGroote Institute for Infectious Disease Research, McMaster University

\vfill\null
\columnbreak

Benjamin M.\ Bolker, PhD

Professor, Departments of Mathematics \& Statistics and Biology, McMaster University

Member, Michael G. DeGroote Institute for Infectious Disease Research, McMaster University

% \vfill\null
\vspace{1em}

Jonathan Dushoff, PhD

Professor, Department of Biology, McMaster University

Member, Michael G. DeGroote Institute for Infectious Disease Research, McMaster University

South African Centre for Epidemiological Modelling and Analysis, University of Stellenbosch, Stellenbosch, South Africa

\end{flushleft}
\end{multicols}

\thispagestyle{empty}

\end{document}
