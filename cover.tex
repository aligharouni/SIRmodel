\documentclass[12pt,letterpaper]{letter}

%% layout
\usepackage[margin=1.25in]{geometry}
\usepackage{multicol}

%% colours
\usepackage{xcolor}
\definecolor{linkcolor}{HTML}{0f87ff} % green
\definecolor{irena}{HTML}{c9657d} % blush
\definecolor{de}{HTML}{5681c4} % blue

\usepackage[colorlinks=true, allcolors=linkcolor]{hyperref}

%% date format
\usepackage{datetime}
\newdateformat{mydate}{\twodigit{\THEDAY}{ }\shortmonthname[\THEMONTH] \THEYEAR}

%% comment macros
\newcommand{\irena}[1]{\textcolor{irena}{$\langle$\textbf{IP:} #1$\rangle$}}
\newcommand{\de}[1]{\textcolor{de}{$\langle$\textbf{DE:} #1$\rangle$}}

%% journal name macro
\newcommand{\journalname}{\emph{BMC Public Health}}

%% data macros
\input{data_macros.tex}

\setlength{\parskip}{0em}

\begin{document}

\begin{multicols}{2}
\footnotesize
\begin{flushleft}

Dr. Natalie Pafitis

Editor, \journalname{}

\vfill

{\normalsize \mydate
\today}
\end{flushleft}

\columnbreak

\begin{flushright}
Irena Papst

Center for Applied Mathematics, Cornell University

657 Rhodes Hall

136 Hoy Rd, Ithaca, NY, USA 14853

\href{mailto:ip98@cornell.edu}{ip98@cornell.edu}
\end{flushright}

\end{multicols}

\setlength{\parskip}{1em}
\thispagestyle{empty}

\vspace{-1em}

Dear Dr.\ Pafitis,

We are pleased to submit an original research article entitled
``Age-dependence of healthcare interventions for COVID-19 in Ontario,
Canada'' for your consideration for publication in \journalname{}. This study specifically aligns with the aims of \journalname{} to publish research related to the social determinants of health and adds new insight into the evolving COVID-19 pandemic.

Age is the most salient clinical indicator of risk from COVID-19 for
most patients. While age-specific distributions of known SARS-CoV-2
infections and COVID-19-related deaths have been widely reported,
relatively few sources present age distributions of hospitalizations
and serious healthcare interventions administered to COVID-19
patients. In our study, we analysed \nccmpresolved{} known SARS-CoV-2 infection
records for the province of Ontario, and quantified the age-intervention associations
for hospitalization, ICU admission, intubations, and ventilations. We
quantified existing age-related impacts of COVID-19 and identified
potential future risks should the healthcare system become overwhelmed
with COVID-19 patients.

Ontario has so far been able to offer and provide COVID-19 treatment
to all that request it. Yet we find that the probability of survival
given hospitalization for COVID-19 decreases substantially after age
40. The distribution of COVID-19-related hospitalizations spans a much
wider age range than deaths, which are concentrated in the extremely
elderly. If prevalence of the virus were to increase in the general
population (as is currently occurring), and ICUs or hospitals were to
subsequently reach maximum capacity, the distribution of deaths might expand toward younger ages.

While our work is specific to the Ontario SARS-CoV-2 epidemic, it has
more general and important implications for COVID-19 outbreaks all
over the world, in terms of both public health and clinical
practice. Our study reveals a potentially serious mortality risk to
middle-aged individuals from SARS-CoV-2 infection if the healthcare
system were to be overwhelmed with COVID-19 patients; this risk is not
captured by the widely available distributions of known SARS-CoV-2
infections and COVID-19-related deaths.

We have presented our study to the Ontario government's COVID-19
science advisory panel (the Ontario Science Table,
\url{https://covid19-sciencetable.ca/}), where it was received with
great interest.  We believe that the readership of \journalname{}
should be made aware of our findings as soon as possible.

All authors have approved the manuscript for submission and declare no competing interests. A preprint of this work has been posted to medRxiv (\url{https://www.medrxiv.org/content/10.1101/2020.09.01.20186395v2}). The manuscript has not been published or submitted for publication elsewhere.

Thank you very much for your consideration,

\begin{multicols}{2}
\begin{flushleft}
\footnotesize

Irena Papst, MSc
\setlength{\parskip}{0em}

PhD Candidate, Center for Applied Mathematics, Cornell University

\vspace{1em}

Michael Li, PhD

Postdoctoral Fellow, Department of Biology, McMaster University

Research Associate, South African Centre for Epidemiological Modelling and Analysis, University of Stellenbosch

\vspace{1em}

David Champredon, PhD

Postdoctoral Associate, Department of Pathology and Laboratory Medicine,

Western University

\vspace{1em}

\vfill\null
\columnbreak

Benjamin M.\ Bolker, PhD

Professor, Departments of Mathematics \& Statistics and Biology, McMaster University

Member, Michael G. DeGroote Institute for Infectious Disease Research, McMaster University

\vspace{1em}

Jonathan Dushoff, PhD

Professor, Department of Biology, McMaster University

Member, Michael G. DeGroote Institute for Infectious Disease Research, McMaster University

\vspace{1em}

David J.\,D.\ Earn, PhD

Professor, Department of Mathematics \& Statistics, McMaster University

Member, Michael G. DeGroote Institute for Infectious Disease Research, McMaster University

Professor, Department of Mathematics, University of Toronto
\end{flushleft}
\end{multicols}

\thispagestyle{empty}

\end{document}
