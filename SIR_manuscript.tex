\documentclass{article}
\usepackage{natbib}
\usepackage{hyperref}
\usepackage{graphicx}
\usepackage{subcaption}
\usepackage{amsmath}
\bibliographystyle{apalike}

\title{SIR Model with Testing and Isolation Mechanisms}

\begin{document}
\maketitle
test $x^2$

% %%%%%%%
\section{Literature Review}

\subsection{Explicit models of TTI (trace/test/isolate) based on network or agent-based models}

% \citep{endo2020implication} [Ali: It seems to me that this is just a statistical model to estimate the parent-offspring of an infected index, not sure if it fits into agent-based group!] Used simulation on a branching process model to assess the forward and backward contact tracing efficiency. Assuming a negative-binomial branching process with a mean R, reproduction number, and overdispersion parameter k, the mean total number of generation G3 and averted G3 are estimated. The effectiveness of TTI is defined as the ratio of averted to the mean.

% %%%%%%%
\bibliography{../SIRlibrary}

\end{document}