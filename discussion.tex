% \section{Discussion}

In this paper, we have developed and analyzed a simple compartmental model that combines epidemiological dynamics---as defined by a simple SIR model---with the dynamics of testing and isolation. Our model is a caricature: it models the most basic feedbacks between epidemic and testing processes, but does not attempt to incorporate the many known complications of COVID-19 epidemiology (e.g., exposed, pre-symptomatic, and asymptomatic compartments \citep{kain2021chopping}; time-varying testing rates; behavioural dynamics \citep{weitz2020awareness}). Thus, it is most appropriate for assessing the \emph{qualitative} phenomena that arise from the interactions between transmission dynamics and testing, rather than for making quantitative predictions or guiding pandemic responses.

Many of the qualitative results we have derived confirm simple, common-sense intuitions. In particular, we can generally decrease $\Rnum$ by increasing isolation efficacy or testing intensity; returning tests faster, if individuals do not isolate while they are waiting for results; or increasing testing focus to target individuals who are likely to be infectious (e.g., symptomatic people or close contacts of known infections).

However, we did find two surprising patterns: under some conditions longer delays in returning tests can reduce epidemic spread, and increasing testing rates can increase spread.

Over broad regions of parameter space, decreasing $\omega$---i.e., \emph{slowing} the rate at which test results are returned---\emph{decreases} $\Rnum$ (for random testing, the parameter region is $\theta\_w \gtrsim 0.25$; for targeted testing, $\theta\_w \geq 0.25$ and either $\theta\_c \ge 0.5$ or $1/\omega > 5$; see Table~\ref{tab:params} for parameter definitions).  This result is counterintuitive and would not be expected by public health authorities that have invested a great deal of effort in reducing delays from testing to results. Dynamically, this effect occurs because speeding up test returns shortens the isolation period of uninfected individuals (for infected people it only shortens the time to progression to the isolation level of the confirmed-positive compartment). Slowing test returns increases $\Rnum$ only if the proportion of infectives in the tested population is high and isolation is relatively strong among people waiting for test results. 

While slowing test returns does decrease $\Rnum$ over broad regions of parameter space in our model, there are several real-world processes missing from our model that make it unlikely that slowing test returns would actually be an effective public health measure. First, we do not model the primary benefit of rapid testing, i.e., detecting and containing outbreaks while they are still in progress. This process could be modeled phenomenologically by making the testing focus more targeted as an increasing proportion of cases is detected, because finding infections allows tests to be concentrated on their connections. Second, individuals may become less likely to maintain isolation if they are required to do so for longer; phenomenologically, we could allow  effectiveness of isolation in the waiting population to be an increasing function of test-return speed, or we could introduce a separate ``waiting, but no longer isolating'' compartment that individuals entered from the ``waiting, isolated'' compartment at a specified rate. Finally, if one wants to decrease the overall transmission rate of the population there are more effective methods than keeping tested people in limbo; these include masking, ventilation, distancing measures, retail and event closures, and stay-at-home orders.

The other counterintuitive result from our analysis is that, for sufficiently high testing intensity $\rho$, further increasing testing intensity can actually \emph{increase} $\Rnum$ (e.g., \fref{pan2}(b), upper right panel [$\theta\_c=1$, $\theta\_w=0$]). This phenomenon can occur because we are considering the DFE in the presence of testing; thus there is an equilibrium distribution of susceptibles between the $S\_n$ (untested) and $S\_u$ (waiting) compartments even as the disease approaches extinction. A higher rate of testing leads to a greater proportion of individuals waiting for negative tests at the DFE. If infected, individuals in this group will take longer to be tested again and to subsequently isolate (because they must wait for their negative tests to be returned before being tested again). If isolation in this group ($\theta\_w$) is low, this effect can under some (relatively rare) circumstances (high $\theta\_c$, low $\omega$, high $\rho$) allow $\Rnum$ to increase with testing intensity. 
We can show that this phenomenon occurs \emph{only} under targeted testing ($w_I > w_S$), but we have not yet found a simple explanation of why it cannot occur under random (unfocused) testing. This phenomenon is also unlikely to occur in the real world. In particular, it depends on levels of testing that are unrealistically high (at least in large, general-population settings).

Although we model the testing process in more detail than typical epidemiological models, one place where more detail could be informative is in the processes determining the testing weights $\{w_S, w_I, w_R\}$. While random testing, as done for surveillance purposes, unambiguously leads to equal testing weights, making precise quantitative connections between public-health practices and testing weights is difficult in other contexts. The testing weights reflect the correlation between an individual's risk of infection and their likelihood of being tested due to age, occupation, geographic location, etc.. This correlation is influenced, among many other factors, by the proportion of the uninfected population with COVID-like symptoms (e.g., due to seasonal upper respiratory tract infections); the concentration of transmission and testing in hot spots such as long-term care facilities and high-density workplaces; the overall testing intensity (and hence, e.g., restriction to symptomatic individuals); and the proportion of COVID-infected people who are symptomatic. Future steps should explore mathematically tractable ways to model some of these factors more precisely. For example, separating the infected class into exposed, symptomatic, and a- or pre-symptomatic compartments and allowing the testing weights to vary across non-symptomatic (exposed/asymptomatic/presymptomatic) vs.\ symptomatic compartments could reflect the allocation of tests for diagnostic purposes (targeting symptomatic individuals) vs.\ contact-tracing (targeting infected but non-symptomatic individuals) vs.\ screening (relatively equal weights, depending on the venue). Alternatively, one could make the testing weights depend on the testing intensity or test-return rate as suggested above. Whatever complexity is added would probably put the model beyond reach of the analytical methods we have used in this paper, but one could still use semi-numerical methods such as constructing the next-generation matrix and using it to evaluate the derivatives of $\Rnum$ with respect to the parameters numerically.

Although testing and tracing is a key part of infection control strategies, mathematical epidemiologists have typically analyzed it with detailed models designed to inform particular public health efforts \citep{endo2020implication,hellewell2020feasibility,jenness2020modeling}, rather than analyzing simple but general models of the feedback between testing and transmission dynamics.
There have been several modeling studies of testing and tracing dynamics and their interaction with epidemiological dynamics. In the context of repeated screening and random testing of isolated populations (such as the members of a university), \cite{bergstrom2020frequency} provided analytical results quantifying the effects that proactive screening of asymptomatic individuals and isolation of confirmed-positive cases could have in reducing the spread of disease. 
\cite{rogers2021high} simulated a SEIR model with testing and isolation; they similarly suggest a strategy of rapid testing with antigen tests and the subsequent isolation of confirmed-positive individuals.
\cite{friston2021testing} model the effects of self-isolation on testing and tracing with a focus on projections under different testing and tracing scenarios. They conclude that the emergence of a second wave depends almost primarily on the rate at which immunity is lost and that it is necessary to track asymptomatic individuals in order to control the outbreak. 
Our modeling approach differs from these previous efforts in that it examines the effects of test-return rates and of different levels of testing focus, from random to highly targeted. We hope this paper will inspire further explorations of the fundamental properties of epidemic models that incorporate explicit testing processes.
