\documentclass[12pt]{article}
\usepackage{natbib}
\usepackage{hyperref}
\usepackage{grffile}
\usepackage{graphicx}
\usepackage{subcaption}
\usepackage{amssymb,amsmath,amsthm}
\usepackage{xcolor}
\usepackage{xspace}
\usepackage[nameinlink,capitalize]{cleveref}
\usepackage{cleveref}
\usepackage[margin=1in]{geometry}
\usepackage{lineno}\renewcommand\thelinenumber{\color{gray}\arabic{linenumber}}
\usepackage{pdflscape}
\usepackage{enumerate}

\newcommand{\todo}[1]{\comment{red}{TODO}{#1}}

\usepackage{xspace}

\newcommand{\fref}[1]{Fig.~\ref{#1}}
\newcommand{\Rlogo}{R\xspace}
\newcommand{\percap}{\emph{per capita}\xspace}
\newcommand{\R}{\ensuremath{\mathcal{R}_0}}
\newcommand{\covid}{COVID-19\xspace}
\newcommand{\pro}[1][]{\ensuremath{\frac{\partial #1}{\partial \rho}}}
\newcommand\pder[2][]{\ensuremath{\frac{\partial#1}{\partial#2}}} %\pder[x]{y}
\renewcommand{\vec}[1]{\ensuremath{\mathbf{#1}}} %bold font for vector

\newcommand{\comment}{\showcomment}
\newcommand{\showcomment}[3]{\textcolor{#1}{\textbf{[#2: }\textsl{#3}\textbf{]}}}
\newcommand{\nocomment}[3]{}

\theoremstyle{definition} % amsthm only
\newtheorem{proposition}{Proposition}
\newtheorem{theorem}{Theorem}

\bibliographystyle{apalike}

\title{The analogy of the cohort-equations paradigm to the compartmental epidemic models}

\begin{document}
\maketitle
\linenumbers
%%%%%%%%%%%%%%%%%%%%
\section{Vision}
\begin{enumerate}
\item 2 frameworks of \cite{van2002reproduction}, I call it compartment framework, and \cite{champredon2018equivalence}, I call it cohort framework, can be tied together and a mechanistic approach to go from one framework to another can be constructed.

\item Having the cohort framework enables one to study the strength-like and speed-like interventions. For example in our SIR model with testing and isolation, testing susciptables is a strength-like intervention and testing the infecteds' is a speed-like intervention. \todo{more context is required here.}
\end{enumerate}

%%%%%%%%%%%%%%%%%%%%
\section{Math foundation}
Notation; we use $I'$ for the cohort-framework of $dI/d\tau$, where $\tau$ is in the infection-time scale and $\dot I$ for the compartment-framework of $dI/dt$. 

The matrix form of the cohort framework consists of the following 3 steps.
\begin{enumerate}[{\it step 1}]
\item
write the cohort Eq. 
\begin{equation}
\label{eq:cohort}
\vec I'=-V \vec I,
\end{equation}
 with the proper initial condition $I(0)$, where the n-by-n matrix $V$ is the flow matrix of leaving infected compartments in \cite{van2002reproduction}.

\item 
finding the intrinsic infectiousness kernel by integrating the cohort Eq. and solving for its time evolution, thus
the solution would be 
\begin{equation}
\label{eq:I}
\vec I(\tau) = \exp(-V\tau) \vec I(0).
\end{equation}
The kernel will be 
\begin{equation}
\label{eq:kernel}
\vec K(\tau) = F \vec I(\tau), 
\end{equation}
where $F$ is the matrix of new infections in the compartment framework.
Note that the n-by-n matrix $\exp(-V\tau)$ is the probability of being infected and stay infectious at time $\tau$. 
% where $\exp(-V\tau)=\sum_{n=0}^\infty \tau^n (-V)^n/n!$.

\item
Calculating $\vec R$ by integrating kernel $\vec K(\tau)$. Note that $\R$ would be the non-negative element of $\vec R$.
\end{enumerate}
Note that in the compartment framework \citep{van2002reproduction}, steps 2 and 3 are combined.
%%%%%%%%%%%%%%%%%%%%
\section{Examples}

\subsection {Simple SIR}
The cohort analogy of a simple SIR compartmental model where $\dot I=\beta S I/N-\gamma I$. The cohort framework is via the cohort Eq. with the following steps.
{\it step 1;} write the cohort Eq. $I'=-\gamma I$ with the initial condition $I(0)=1$, 
{\it step 2;} finding the intrinsic infectiousness kernel by integrating the cohort Eq. and solving for its time evolution, thus $k(\tau)=\beta \exp(-\gamma \tau)$.
{\it step 3;} $\R$ will be the integration of the kernel. That is $\R=\beta/\gamma$. 

\subsection{Simple SEIR model}
The simplest form of a SEIR model without vital rates is 
\begin{align}\label{mo:sier1}
\dot S=-\beta\,S\,I/N,~~
\dot E= \beta\,S\,I/N -\sigma\,E,~~
\dot I= \sigma\,E -\gamma\,I,~~
\dot R= \gamma\,I.
\end{align}
We know that $\R=\beta/\gamma$ from next generation method where 

\begin{align}
\label{seir1FV}
F = \beta \left[ \begin {array}{cc} 
0 & 1 \\
0 & 0
 \end {array} \right],~~
V=
 \left[ \begin {array}{cc}
 \sigma & 0 \\
-\sigma & \gamma
\end {array} \right],
~~ \text{thus}~~
V^{-1} = \frac{1}{\sigma\,\gamma}
\left[ \begin {array}{cc}
\gamma & 0 \\
\sigma & \sigma
\end {array} \right].
\end{align}
I the context of the cohort framework we show that the same $\R$ can be calculated by following the three steps.

{\it Step 1;} The cohort equation with the initial condition (in matrix form) is 
\begin{align}
\vec I'=-V\,\vec I,~~ \text{where}~~ \vec I(\tau)=\left[ \begin {array}{c} E(\tau)\\I(\tau) \end {array} \right], ~~
\vec I(0)=\left[ \begin {array}{c} 1\\0 \end {array} \right].
\end{align}
Solving for $I(\tau)$ by finding the eigenvalues and eigenvectors of the above linear system results in
\begin{align}
\vec I(\tau) = \left[ \begin {array}{c} E(\tau)\\I(\tau) \end {array} \right] =
c_1 \exp(-\sigma\,\tau)\,\left[ \begin {array}{c} \gamma/\sigma-1\\1 \end {array} \right] 
-c_1 \exp(-\gamma\,\tau)\,\left[ \begin {array}{c} 0\\1 \end {array} \right],
\end{align}
where $$c_1=\frac{\sigma}{\gamma-\sigma}.$$

{\it Step 2;} The kernel is
\begin{align}
\vec K(\tau)= F \vec I(\tau)
= c_1 \exp(-\sigma\,\tau)\,\left[ \begin {array}{c} \beta\\0 \end {array} \right] 
 -c_1 \exp(-\gamma\,\tau)\,\left[ \begin {array}{c} \beta\\0 \end {array} \right].
\end{align}

{\it Step 3;} The basic reproduction number $\R$ is given by integrating the non-zero element of the kernel as follows.
\begin{align}
\R=\int_0^\infty \vec K_1(\tau) d\tau
=\beta\,c_1(1/\sigma-1/\gamma)
=\beta/\gamma.
\end{align}



%%%%%%%%%%%%%%%%%%%%

\bibliography{SIRlibrary}
\end{document}