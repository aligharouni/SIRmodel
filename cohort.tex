\documentclass[12pt]{article}
\usepackage{natbib}
\usepackage{hyperref}
\usepackage{grffile}
\usepackage{graphicx}
\usepackage{subcaption}
\usepackage{amssymb,amsmath,amsthm}
\usepackage{xcolor}
\usepackage{xspace}
\usepackage[nameinlink,capitalize]{cleveref}
\usepackage{cleveref}
\usepackage[margin=1in]{geometry}
\usepackage{lineno}\renewcommand\thelinenumber{\color{gray}\arabic{linenumber}}
\usepackage{pdflscape}
\usepackage{enumerate}

\newcommand{\todo}[1]{\comment{red}{TODO}{#1}}

\usepackage{xspace}

\newcommand{\fref}[1]{Fig.~\ref{#1}}
\newcommand{\Rlogo}{R\xspace}
\newcommand{\percap}{\emph{per capita}\xspace}
\newcommand{\Rnot}{\ensuremath{\mathcal{R}_0}}
\newcommand{\covid}{COVID-19\xspace}
\newcommand{\pro}[1][]{\ensuremath{\frac{\partial #1}{\partial \rho}}}
\newcommand\pder[2][]{\ensuremath{\frac{\partial#1}{\partial#2}}} %\pder[x]{y}

\newcommand{\comment}{\showcomment}
\newcommand{\showcomment}[3]{\textcolor{#1}{\textbf{[#2: }\textsl{#3}\textbf{]}}}
\newcommand{\nocomment}[3]{}

\theoremstyle{definition} % amsthm only
\newtheorem{proposition}{Proposition}
\newtheorem{theorem}{Theorem}

\bibliographystyle{apalike}

\title{The analogy of the cohort-equations paradigm to the compartmental epidemic models}

\begin{document}
\maketitle
\linenumbers
%%%%%%%%%%%%%%%%%%%%
\section{Vision}
\begin{enumerate}
\item 2 frameworks of \cite{van2002reproduction}, I call it compartment framework, and \cite{champredon2018equivalence}, I call it cohort framework, can be tied together and a mechanistic approach to go from one framework to another can be constructed.

\item Having the cohort framework enables one to study the strength-like and speed-like interventions. For example in our SIR model with testing and isolation, testing susciptables is a strength-like intervention and testing the infecteds' is a speed-like intervention. \todo{more context is required here.}
\end{enumerate}

%%%%%%%%%%%%%%%%%%%%
\section{Math foundation}
Notation; we use $I'$ for the cohort-framework of $dI/d\tau$, where $\tau$ is in the infection-time scale and $\dot I$ for the compartment-framework of $dI/dt$. 

The matrix form of the cohort framework consists of the following 3 steps.
\begin{enumerate}[step 1]
\item
write the cohort Eq. 
\begin{equation}
\label{eq:cohort}
I'=-V I,
\end{equation}
 with the proper initial condition $I(0)$, where the n-by-n matrix $V$ is the flow matrix of leaving infected compartments in \cite{van2002reproduction}.

\item 
finding the intrinsic infectiousness kernel by integrating the cohort Eq. and solving for its time evolution, thus
the solution would be 
\begin{equation}
\label{eq:I}
I(\tau) = \exp(-V\tau) I(0).
\end{equation}
The kernel will be 
\begin{equation}
\label{eq:kernel}
K(\tau) = F I(\tau), 
\end{equation}
where $F$ is the matrix of new infections in the compartment framework.
Note that the n-by-n matrix $\exp(-V\tau)$ is the probability of being infected and stay infectious at time $\tau$, where
$\exp(-V\tau)=\sum_{n=0}^\infty \tau^n (-V)^n/n!$.

\item
Calculating $R$ by integrating kernel $K(\tau)$.
\end{enumerate}

Note that in the comartment framework, steps 2 and 3 are combined.
%%%%%%%%%%%%%%%%%%%%
\section{examples}

{\bf Example 1, the simple SIR;}
The cohort analogy of a simple SIR compartmental model where $\dot I=\beta S I/N-\gamma I$. The cohort framework is via the cohort Eq. with the following steps
step 1: write the cohort Eq. $I'=-\gamma I$ with the initial condition $I(0)=1$, 

step 2: finding the intrinsic infectiousness kernel by integrating the cohort Eq. and solving for its time evolution, thus
$k(\tau)=\beta \exp(-\gamma \tau)$.

step 3: $R$ will be the integration of the kernel. That is $R=\beta/\gamma$. 

{\bf Example 2, the SIR model with testing;}
In the context of our SIR model with testing, $I^T=[I_u,I_n,I_p,I_c]$ where $T$ is the transposed operator.
And the solution is given by $I(\tau) = \exp(-V\tau) I(0)$ where $I(0)=[1,0,0,0]$.


%%%%%%%%%%%%%%%%%%%%

\bibliography{SIRlibrary}
\end{document}