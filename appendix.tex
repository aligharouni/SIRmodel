
\setcounter{equation}{0}
\setcounter{figure}{0}
\setcounter{table}{0}
% \setcounter{page}{1}
\makeatletter
\renewcommand{\theequation}{A\arabic{equation}}
\renewcommand{\thefigure}{A\arabic{figure}}
\renewcommand{\bibnumfmt}[1]{[A#1]}
\renewcommand{\citenumfont}[1]{A#1}

\subsection{Model and calculation of $\Rnum$}\label{app:R0}

The model in the form of a system of ordinary differential equations is 
\begin{subequations}\label{model}
\begin{align}
 d S\_u/dt &= -\Lambda S\_u - \testing{S} S\_u + \omega S\_n, \\
 d S\_n/dt &= -(1-\theta\_w)\Lambda S\_n + (1-p_S) \testing{S} S\_u - \omega S\_n, \\
 d S\_p/dt &= -(1-\theta\_w)\Lambda S\_p + p_S \testing{S} S\_u - \omega S\_p, \\
 d S\_c/dt &= -(1-\theta\_c)\Lambda S\_c + \omega S\_p, \\
 d I\_u/dt &= \Lambda S\_u - \testing{I} I\_u + \omega I\_n  - \gamma I\_u, \\
 d I\_n/dt &= (1-\theta\_w)\Lambda S\_n + (1-p_I) \testing{I} I\_u - \omega I\_n -\gamma I\_n, \\
 d I\_p/dt &= (1-\theta\_w)\Lambda S\_p + p_I \testing{I} I\_u - \omega I\_p -\gamma I\_p, \\
 d I\_c/dt &= (1-\theta\_c)\Lambda S\_c + \omega I\_p - \gamma I\_c, \\
 d R\_u/dt &= \gamma I\_u - \testing{R} R\_u + \omega R\_n, \\
 d R\_n/dt &= \gamma I\_n + (1-p_R) \testing{R} R\_u - \omega R\_n, \\
 d R\_p/dt &= \gamma I\_p + p_R \testing{R} R\_u  - \omega R\_p, \\
 d R\_c/dt&= \gamma I\_c + \omega R\_p, \\
 dN/dt &= \omega (S\_n + I\_n + R\_n),  \\
 dP/dt &= \omega(I\_p + R\_p) ,
\end{align}
\end{subequations}
%
(see Table \ref{tab:params} for parameter definitions). The next generation matrix for this model is $G = F V^{-1}$, where matrix $F$ represents the inflow of new infection to the infected compartments and matrix $V$ represents the flow in the infected compartments when the population is totally susceptible. 
Matrices $F$ and $V$ are
\begin{align}
\label{F}
F =& \beta/N \left[ \begin {array}{cccc} 
S\_u^* & (1-\theta\_w) S\_u^* & (1-\theta\_w) S\_u^* & (1-\theta\_c) S\_u^*\\
(1-\theta\_w)S\_n^* & (1-\theta\_w)^2 S\_n^* & (1-\theta\_w)^2 S\_n^* & (1-\theta\_w)(1-\theta\_c) S\_n^* \\ 
0&0&0&0\\
0&0&0&0
 \end {array} \right] \\ \notag
 =&\beta/N \left[ \begin {array}{c} S\_u^* \\ (1-\theta\_w)S\_n^* \\ 0 \\ 0 \end {array} \right]
        \left[ \begin {array}{cccc} 1,   1-\theta\_w,   1-\theta\_w,   1-\theta\_c \end {array} \right],
\end{align}
and 
\begin{align}
\label{V}
V= \left[ \begin {array}{cccc}  
\testinghat{I}+\gamma&-\omega&0&0\\
-(1-p_I)\testinghat{I}&\omega+\gamma&0&0\\
-p_I \testinghat{I}&0&\omega+\gamma&0\\
0&0&-\omega&\gamma
\end {array} \right].
\end{align}
The matrix inverse of $V$ is 
\begin{align}
\label{vinv}
V^{-1} =
\frac{1}{\gamma C}
\left[ \begin {array}{cccc}
\gamma(\omega+\gamma)^2 & \gamma\omega(\omega+\gamma)&0&0\\ \noalign{\medskip}
\gamma(\omega+\gamma)(1-p_I) \testinghat{I} & \gamma(\omega+\gamma)(\testinghat{I}+\gamma)&0&0 \\ \noalign{\medskip}
\gamma(\omega+\gamma) p_I \testinghat{I} & \gamma\omega p_I \testinghat{I} & C \gamma/(\omega+\gamma) & 0 \\ \noalign{\medskip}
\omega(\omega+\gamma) p_I \testinghat{I} & \omega^2 p_I \testinghat{I} & C \omega/(\omega+\gamma) & C
\end {array} \right],
\end{align}
where $C= \big( \gamma(\omega+\gamma)+(\gamma+\omega p_I)\testinghat{I} \big) (\omega+\gamma)$ and $\testinghat{I}$ is the \percap testing rate for the infected people and represented in Eq.~\eqref{eq:fi}. Note that all the columns of matrix $V^{-1}$ summ up to $1/\gamma$.

The particular form of $F$ with two rows of zeros at the bottom results in the following blocked form of matrix $G$.
\begin{equation}
\label{mat:G}
G = \left[ \begin {array}{cc}
G_{11}&G_{12}\\
0&0
\end {array} \right],
\end{equation}
where both blocked matrices $G_{11}$ and $G_{12}$ are 2 by 2. Given the upper triangular form of matrix $G$, the basic reproduction number $\Rnum$ (defined as the spectral radius of matrix $G$) is only determined by the blocked matrix $G_{11}$,
\begin{equation}
\label{G11}
G_{11} = \frac{\beta}{\gamma   C} 
\left[ \begin {array}{c} (\omega-\rho)/\omega \\ (1-\theta\_w)\rho/\omega \end {array} \right]
\left[ \begin {array}{cccc} 1,   1-\theta\_w,   1-\theta\_w,   1-\theta\_c \end {array} \right]
\left[ \begin {array}{cc}
\gamma(\omega+\gamma)^2 & \gamma\omega(\omega+\gamma)\\ \noalign{\medskip}
\gamma(\omega+\gamma)(1-p_I)\testinghat{I} & \gamma(\omega+\gamma)(\testinghat{I}+\gamma) \\ \noalign{\medskip}
\gamma(\omega+\gamma) p_I \testinghat{I} & \gamma\omega p_I \testinghat{I} \\ \noalign{\medskip}
\omega(\omega+\gamma) p_I \testinghat{I} & \omega^2 p_I \testinghat{I}
\end {array} \right].
\end{equation}
It is notable that matrix $F$ \eqref{F} has rank one and consequently $G_{11}$ does so. That is $G_{11}$ has only one non-zero eigenvalue which is $\Rnum$.

The expression of $\Rnum$ has a complicated form with all of the model parameters involved. This expression can be simplified and represented given the specific form of matrix $G_{11}$ \eqref{G11}. For the purpose of simplicity we present $\Rnum$ in the manuscript in terms of expressions $C$, $C_1$ and $C_2$, specified in \eqref{eq:C12}. 

It remains difficult to show that the reproduction number $\Rnum$ is decreasing with respect to \percap testing intensity, $\rho$, and the speed of the test return, $\omega$, for the feasible ranges of the parameters, that is
\begin{align}
\label{cond}
& \omega>0, \\
& 0 \leq \rho<\omega,\\ 
& 0 \leq \theta\_w \leq \theta\_c \leq 1, \\
& \oldtesttarget\geq 1.
\end{align}
In realistic cases the testing rate $\rho$ is very small (i.e., only a small fraction of the population can be tested every day); it is thus reasonable to use a linear approximation of $\Rnum$ for $\rho \ll 1$ to analyze the behaviour of $\Rnum$ with respect to $\omega$ (see section \appref{omega}). 
In the next section we provide an equivalent representation of $\Rnum$ in order to show that increasing testing intensity typically decreases $\Rnum$.

% %%%%%%%
\subsection{More testing intensity may decrease $\Rnum$}\label{app:rho}

This section shows that $\pro{\Delta}$ can be positive or negative, with $\Delta$ defined in Eq.~\eqref{eq:C12}, and thus $\pro{\Rnum} < 0$, where $\Rnum$ is given in Eq.~\eqref{R0}. We rewrite matrix $G_{11}$ in \eqref{G11} in the following form to simplify the calculations:
\begin{equation}
\label{G112}
G_{11} = \frac{\beta}{\gamma C} 
\left[ \begin {array}{c}  S\_u^*/N \\ (1-\theta\_w) S\_n^*/N  \end {array} \right]
\left[ \begin {array}{cccc} 
C-C_1, C-C_2\end {array} \right],
\end{equation}
where $C$ is the same as the one in Eq. \eqref{eq:C12}, i.e., 
$$C=(\omega+\gamma)(\gamma(\omega+\gamma)+(\omega p_I+\gamma) \testinghat{I}),$$
and $C_1$ and $C_2$ are 
\begin{align*}
% \label{eq:C12}
C_1 =& (\omega+\gamma)(\theta\_w   \gamma+\theta\_c  \omega  p_I) \testinghat{I},\\
C_2 =& \big( \omega\gamma(1+p_I)\testinghat{I}+\gamma^2 (\omega+\gamma+\testinghat{I})\big)\theta\_w + \omega^2 p_I \testinghat{I} \theta\_c,
\end{align*}
where $\testinghat{I}$ is given in Eq.~\eqref{eq:fi}.
Note that for analysis brevity, we let $N=1$, thus $S\_u^*$ and $S\_n^*$ are in the scale of 0 to 1.
$\Rnum$ is in the same form as in Eq. \eqref{R0}  
$$\Rnum= \frac{\beta}{\gamma} (1-\Delta),$$
where 
\begin{equation*}
\label{eq:del4}
\Delta= \frac{1}{C}\big(C_1 S\_u^*+(C_2(1-\theta\_w)+C \theta\_w) S\_n^*\big).
\end{equation*}

The first goal
is to explore how changes in isolation, $\theta\_w$ and $\theta\_c$, affects $\Rnum$. Mathematically we would like to verify the sign of $\pder \Rnum{\theta\_w}$ and $\pder \Rnum{\theta\_c}$. We start with simplifying $\Delta$ \eqref{eq:del4} by factoring $\theta\_w$ and $\theta\_c$ in Eq.~\eqref{eq:del4}. Thus, $\Delta$ can be rewritten as
\begin{equation}
\label{eq:del4_theta}
\Delta= \frac{1}{C}\Big(
-D_1 S\_n^* \theta\_w^2 +
\big(-\omega^2 p_I \testinghat{I} S\_n^*\theta\_c+ D_2 S\_n^*+\gamma \testinghat{I}(\omega+\gamma) \big)\theta\_w+
(\omega+\gamma S\_u^*)\omega p_I \testinghat{I} \theta\_c
\Big),
\end{equation}
where
\begin{align}
\label{eq:del4_d1d2}
D_1=& (\omega+\gamma)\gamma^2+(\omega+\gamma+\omega p_I)\gamma \testinghat{I}, \\
D_2=& (3\omega+2\gamma)\gamma^2+(\omega+\gamma+2\omega p_I)\gamma \testinghat{I}+(\gamma+p_I \testinghat{I})\omega^2.
\end{align}
$\Delta$, Eq.~\eqref{eq:del4_theta}, is linear in $\theta\_c$ with a positive coefficient. thus
\begin{equation}
\label{eq:d_del4_thetac}
\pder\Delta{\theta\_c}=1/C(\gamma S\_u^*+\omega(1-\theta\_w S\_n^*))\omega p_I \testinghat{I}.
\end{equation}
This results in increasing $\Delta$, thus decreasing $\Rnum$ with respect to $\theta\_c$, that is $\pder \Rnum{\theta\_c}\leq 0$. Note that $C$ is independent of $\theta\_c$ and $\theta\_w$. 

With a similar logic, $\Delta$ \eqref{eq:del4_theta} is a concave-down quadratic equation in $\theta\_w$, given by
\begin{equation}
\label{eq:del4_thetaw}
1/C \Big( -D_1 S\_n^*\theta\_w^2+
\big(-\omega^2 p_I \testinghat{I} S\_n^*\theta\_c+D_2 S\_n^*+\gamma \testinghat{I}(\omega+\gamma) \big)\theta\_w\Big).
\end{equation}
We show that the feasible range of $\theta\_w$ lies between 0 and the vertex of this parabola where the parabola is increasing in $\theta\_w$, and so does $\Delta$ which results in inferring $\pder\Rnum{\theta\_w}\leq 0$.
It is enough to show that partial derivative of the expression \eqref{eq:del4_thetaw} with respect to $\theta\_w$ at $\theta\_w=1$ is non-negative. It follows that
\begin{align}
\label{eq:d_del4_thetaw}
\pder\Delta{\theta\_w}\bigg\rvert_{\theta\_w=1} =&
1/C\Big( (D_2-2D_1-\omega^2 p_I \testinghat{I} \theta\_c) S\_n^*+\gamma \testinghat{I}(\omega+\gamma) \Big) \\\notag
=& 1/C\Big( (\gamma(\omega+\gamma)+\gamma\omega^2+(1-\theta\_c)\omega^2 p_I \testinghat{I} ) S\_n^*
+\gamma(\omega+\gamma) \testinghat{I} (1-S\_n^*) \Big),
\end{align}
which is a positive quantity, given that $\theta\_c$ and $S\_n^*$ vary between 0 and 1.

The second goal is to explore how changes in \percap testing intensity $\rho$ affects $\Rnum$. Mathematically we would like to verify the sign of $\pder\Rnum{\rho}$, which specifically depends on $\pder\Delta{\rho}$. We use the derived expressions for $S\_u^*$ and $S\_n^*$, given by Eqs.~\eqref{dfe}, in $\Delta$ \eqref{eq:del4}. Also, we define $\phi = \testinghat{S} = \frac{\rho \omega}{\omega-\rho}$, to reparameterize $\rho$. This is mainly to avoid singularity in $\testinghat{I}$ \eqref{eq:fi}, when testing intensity $\rho$ is very close to the rate of test return $\omega$. Thus, $\rho$ is reparameterized as 
\begin{equation}
\label{eq:phi}
\rho=\frac{\omega \phi}{\omega+\phi}.
\end{equation}
This one-to-one monotonic reparameterization enables us to simplify the mathematical expressions and explore the simpler $\pder\Delta{\phi}$ instead of the complicated $\pder\Delta{\rho}$.
Defining $\testtarget \equiv \oldtesttarget$, the derivative is 
\begin{align}
\label{eq:dd3dr}
\partial\Delta/\partial\phi=& \frac{1}{d_3} (a_3 \phi^2+b_3 \phi+c_3),
\end{align}
where
\begin{align}
\label{eq:abcd2}
a_3& = \testtarget \Bigg(
&& (1-\theta\_w) (1+\testtarget)\theta\_w \gamma^3 +(1-\theta\_c) p_I^2 \theta\_w \testtarget \omega^3 \\ \notag
& ~ &&+\Big( \big( (1-\theta\_w-\testtarget) \theta\_c +(3-2\theta\_w) \theta\_w\testtarget \big) p_I +(1-\theta\_w) (1+\testtarget)\theta\_w  \Big) \omega \gamma^2 \\ \notag
& ~ &&+\Big(
\big((1-\theta\_w-\theta\_w \testtarget) \theta\_c+ (2-\theta\_w) \theta\_w \testtarget \big) p_I
+(2 \theta\_w-\theta\_w^2-\theta\_c) \testtarget p_I^2
\Big) \omega^2 \gamma \Bigg), \\ \notag
b_3& = && \hspace{-1cm} 2\testtarget (\omega+\gamma)\gamma\Big(
(\omega+\gamma+\omega p_I)(2-\theta\_w) \gamma \theta\_w+ (1-\theta\_w) \omega^2 p_I\theta\_c+\omega^2 p_I\theta\_w
\Big) ,\\ \notag
c_3& = && \hspace{-1cm} (\omega+\gamma)^2 \gamma\Big(
(2-\theta\_w) \gamma^2 \theta\_w+
(1+\testtarget) \omega \gamma \theta\_w+
\testtarget \omega^2 p_I\theta\_c \Big),\\ \notag
d_3&  = && \hspace{-1cm} \frac{(\omega+\gamma)}{\omega}  \Big((\omega p_I+\gamma)\testtarget\phi + (\omega+\gamma)\gamma \Big)^2(\omega+\phi)^2.
\end{align}
Note that $\phi\geq 0$, also $b_3$, $c_3$ and $d_3$ are all positive. However $a_3$ can be positive or negative.
If $a_3\geq 0$, $\partial\Delta/\partial\phi \geq 0$ for all feasible range of parameters, thus $\pro\Rnum \leq 0$. It is straightforward to show that $a_3\geq 0$ when testing is random, i.e., $w\_S=w\_I=1$. 
If $a_3 < 0$, then the quadratic expression in the numerator of \eqref{eq:dd3dr} has a positive root, $\phi^*$, such that for $\phi>\phi^*$, $\partial\Delta/\partial\phi < 0$. 

An example of this countervailing effect of $\phi$, and consequently $\rho$, on $\Rnum$ occurs when $\theta\_w=0$ and $\theta\_c=1$.
This is illustrated in the top-right panel of the \fref{pan2} panel (b), where the strength of isolation for awaiting people is the least, but the most for the confirmed cases. In this case, simplifying $a_3$ in Eq.~\eqref{eq:abcd2} gives
\begin{equation}
  \begin{split}
    a_3 & =\testtarget \omega \gamma p_I\left((\omega+\gamma)-\testtarget(\omega p_I+\gamma)\right)  \\
    & \propto  \omega \left( 1- \testtarget p_I \right) + \gamma \left( 1- \testtarget \right)  .
  \end{split}
\end{equation}
If $p_I>0$, then $a_3<0$ for sufficiently targeted testing (i.e. when $\testtarget p_I >1$; because
$p_I \leq 1$, \testtarget is always $\geq \testtarget p_I$).
%% this LOOKS like we can get a3<0 even if pI=0 for targeting > 1 and sufficiently large gamma/omega,
%% but that seems crazy so I'm probably missing something
When the test is perfectly sensitive ($p_I=1$), $a_3<0$ as long as $\testtarget>1$.
Under either of these conditions, there exists a range for $\rho$ over which $\pder\Rnum{\rho}\leq 0$.  
Because increasing values of $\rho$ and $\omega$ both delay the rate at which individuals flow to the $I\_c$ compartment, it is reasonable that increasing either value could (under appropriate circumstances) increase $\Rnum$.

% %%%%%%%
\subsection{Rate of returning tests} \label{app:omega}
The third goal is to explore how changes in the rate of test return $\omega$ affect $\Rnum$. Mathematically we would like to verify the sign of $\pder\Rnum{\omega}$, which specifically depends on $\pder\Delta{\omega}$. We use
the linearization of $\Delta$ when $\rho \ll 1$ to show that there a non-monotonic relationship between $\Rnum$  and $\omega$. The linear term in the Taylor expansion of $\Delta$ when $\rho \ll 1$ is
\begin{equation}
\label{eq:lin}
\Delta = \frac{\rho}{\omega \gamma(\omega+\gamma)} \Big(
\testtarget \omega^2 p_I \theta\_c+(\testtarget+1) \gamma\omega\theta\_w +\gamma^2 \theta\_w(2-\theta\_w)
\Big). 
\end{equation}
This results in
\begin{equation}
\label{eq:dlindo}
\pder\Delta{\omega} = \frac{\rho}{\omega^2 (\omega+\gamma)^2} \Big(
(p_I \testtarget\theta\_c -(1+\testtarget)\theta\_w)\omega^2-2\theta\_w \gamma(2-\theta\_w) \omega+\theta\_w \gamma^2 (\theta\_w-2)
\Big).
\end{equation}
% ALi & Faddy's stuff
The latter expression has two roots
\begin{align}
\label{eq:omega_roots}
\omega^*_{+} &= \frac{\gamma\Big(-c_4 + \sqrt{c_4^2 + (a_4-b_4)c_4}\Big)}{b_4-a_4} \\
\omega^*_{-} &= \frac{\gamma\Big(-c_4 - \sqrt{c_4^2 + (a_4-b_4)c_4}\Big)}{b_4-a_4},
\end{align}
where $a_4 = p_I\testtarget\theta\_c$, $b_4 = (1+\testtarget)\theta\_w$, $c_4 = \theta\_w(2-\theta\_w)$. This enables us to describe the behaviour of $\Rnum$ with respect to $\omega$ in the following two cases.
\begin{itemize}
\item \textbf{Case I:} If $b_4 \geq a_4$, $\pder\Delta{\omega} < 0$, so $\Rnum$ is always increasing with respect to $\omega$ (i.e., it is always harmful to return tests more rapidly). 
\item \textbf{Case II:} If $b_4 < a_4$, $\Rnum$ will be decreasing with respect to $\omega$ (i.e., returning tests more rapidly is beneficial) only when $\omega > \omega^*_-.$
\end{itemize}
Note that $b_4 \geq a_4$ is characterized by
\begin{equation}\label{case_1_ineq}
\Big(\testtarget+1\Big)\theta\_w \geq \testtarget p_I\theta\_c \Longleftrightarrow \Big(\frac{w_S}{w_I}+1\Big)\frac{1}{p_I} \geq \frac{\theta\_c}{\theta\_w}.
\end{equation}
We begin with a \textbf{proof of Case I}. Suppose that $b_4 \geq a_4$. If the roots of $\pder\Delta{\omega}$ are not complex, then we must have $c_4 \geq b_4-a_4$. Note that $\omega^*_{-}$ must be negative since the numerator is clearly negative but the denominator is positive. Next, note that since $c_4 \geq b_4-a_4$, the numerator of $\omega^*_{+}$ must also be negative, so $\omega^*_{+}$ is negative. Thus, in this case, $\pder\Rnum{\omega}$ does not change sign on $(0,\infty)$. Checking the sign of \ref{eq:dlindo} for arbitrarily large $\omega$ shows that it is negative (since the $(a_4-b_4)\omega^2$ term dominates and is negative). So $\Rnum$ is increasing with respect to $\omega$ on all of $(0,\infty)$.

We now turn our attention to a \textbf{proof of Case II}. Suppose that $b_4 < a_4$. It follows that $\omega^*_+$ and $\omega^*_-$ are real since $c_4 > 0 > b_4-a_4$. Next, note that $\omega^*_-$ is positive since both the numerator and denominator are negative. On the other hand, $\omega^*_+$ is negative since the denominator is negative but the numerator is positive (because $c_4 > b_4-a_4$). Thus, our task is to understand the sign of $\pder\Delta{\omega}$ around the root $\omega^*_-$. Checking the sign of \ref{eq:dlindo} for arbitrarily large $\omega$ shows that it is positive (since the $(a_4-b_4)\omega^2$ term dominates and is positive). Likewise, checking the sign for values of $\omega$ close to $0$ shows that it is negative. Thus, $\Rnum$ is increasing with respect to $\omega$ when $\omega < \omega^*_-$, and is decreasing $\omega > \omega^*_-$. 

Having presented the formal analysis, we now concern ourselves with its biological interpretations. We begin by interpreting \ref{case_1_ineq}, under which returning tests more rapidly is \emph{always} harmful. Notice that the ratio $\frac{\theta\_c}{\theta\_w}$ is simply a measure of how much more strongly individuals self-isolate when they test positive compared to when waiting for tests. Since the rate of test return directly influences the rate at which individuals change from a waiting state to a confirmed-positive state, it is intuitive that $\frac{\theta\_c}{\theta\_w}$ would appear in \ref{case_1_ineq}. Next, note that the left-hand side increases when test sensitivity decreases and when targeting of positive individuals is poor. This is consistent with our intuition: a false negative that is returned more rapidly will allow an infectious individual to relax their self-isolation, thus increasing transmission. Likewise, if individuals tested are mainly susceptible (rather than infectious), then returning tests more slowly would encourage them to self-isolate for longer while awaiting test results. Having understood the role of each of the parameters in \ref{case_1_ineq}, a holistic interpretation of this inequality is that returning tests more slowly is helpful when the benefit of extended self-isolation by infected individuals awaiting test results outweighs the benefit of identifying positive cases. 

Now we interpret \textbf{Case II}.  In this case, $\Rnum$ will have a global maximum with respect to $\omega$ at $\omega^*_-$. Note that our model assumption that $\rho < \omega$ plays an important role here: if $\omega^*_- < \rho$, then $\Rnum$ will be always decreasing with respect to $\omega$. On the other hand, if $\rho < \omega^*_-$, then $\Rnum$ will be increasing with respect to $\omega$ on $(\rho, \omega^*_-)$ and decreasing beyond that.

% %%%%%%%
\subsection{The effect of testing focus parameter $\testtarget$ on $\Rnum$} \label{app:w}

\begin{align}
\label{eq:d_del_wis}
\pder \Delta{w\_{IS}}= \dfrac{(\omega-\rho) (\omega(\omega-\rho\theta\_w)+\gamma(\omega-\rho)) (\theta\_w\gamma+\theta\_c\omega p\_I)}
{(-\omega^2 \gamma+\omega \gamma \rho-\gamma \rho \omega w\_{IS}-
\omega \gamma^2+\gamma^2 \rho-\omega^2 p_I \rho w\_{IS})^2},
\end{align}
which is a positive quantity. Thus, $\pder \Rnum{w\_{IS}}\leq 0$. Therefore, increasing the focus of testing on the infectious people will result in less transmission. 

% %%%%%%%
\subsection{On Testing Rate and Numerical Singularity} \label{app:singularity}

In this work, we do not present any numerical solutions of the ODEs to investigate typical trajectories.
However, attempting to do so quickly reveals a problem in the way the model is posed;
with the scaling of testing ($\sigma$) defined as in the body of the paper (Eq.~\eqref{sigma}), the population in the $S$ compartments appeared to blow up when the system is near the DFE. This occurs because once the only untested people are susceptibles, the FOI approaches $\Lambda=0$, and the testing rate $\testing{S} \to \rho N/S\_u$. Thus, the first equation of the model \eqref{model} will become
$d S\_u/dt = - \rho N + \omega S\_n$. Thus changes in $S\_u$ will be independent on $S\_u$, and the decay of the $S\_u$
population becomes linear rather than exponential --- allowing $S\_u$ to become negative!
To avoid this problem the testing rate, $\sigma$, should be formulated such that people from the untested compartments will not be tested if they are not there.
One way to fix this issue, is to consider a maximum testing rate, $\tau$ (1/day). In general, we want to test at a rate of $\rho$ across the whole population. This won't always be possible, so we impose a maximum rate of $\tau$ per testable person and redefine $\sigma = \frac{\tau \rho N}{\tau W + \rho N}$, with the assumption that $\tau \gg \rho$. This modification of $\sigma$ does not affect any of the results we have derived about the invasion of the epidemic from the DFE (i.e., results on $\Rnum$ and $\Delta$).
