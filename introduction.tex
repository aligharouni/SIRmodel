% \section{Introduction}

The observed dynamics of the \covid epidemic have been driven both by epidemiological processes (infection and recovery) and by testing processes (testing and test reporting). In addition to shaping epidemic observations (via case reports), testing processes also alter epidemiological dynamics \citep{peto2020covid,taipale2020population}. Because individuals with confirmed infections (positive tests) are likely to self-isolate, and individuals who are awaiting the results of a test may also do so, testing will generally increase the number of people who are isolating and hence reduce epidemic growth rates. We developed a mechanistic model that incorporates epidemic processes and testing in order to explore the effects of testing and isolation on epidemic dynamics.

If testing influences behaviour, then epidemic dynamics will depend on who gets tested.
The impacts of testing will depend both on testing intensity (tests performed per day) and on how strongly testing is focused on people who are infectious.
This level of focus depends in turn on the purpose and design of testing programs. 
When testing is done for the purposes of disease surveillance \citep{foddai2020base}
tests are typically conducted randomly (or using a stratified random design) across the population in order to make an unbiased assessment of population prevalence.

Over the course of the \covid pandemic, however, the vast majority of testing has been done with other goals --
primarily diagnostic (determining infection status for clinical purposes) \citep{phua2020intensive,who2020global}, or for control (determining  infection status in order to isolate cases that have been found by contact tracing) \citep{aleta2020modelling,kucharski2020effectiveness,grassly2020comparison,smith2020adherence}, which we characterize as \emph{targeted} testing strategies.
In these situations, testing probabilities can differ sharply across epidemiological compartments; in our dynamical model, we will characterize these probabilities by assigning a testing weight to each compartment that determines the \emph{relative} probability that an individual in that compartment will be selected for testing (see Methods). 

Diagnostic testing focuses on people with infection-like symptoms; thus the relative testing weights for infected people will depend on the relative probability of infected people having symptoms. For \covid infection, the testing weights will depend on the proportion of asymptomatic infections, the time spent pre-symptomatic vs.\ symptomatic during the course of an infection, and on the incidence of \covid-like symptoms among people in the population \emph{not} infected with \covid. Testing for epidemic control focuses on known contacts of infected people; in this case the testing weights for infected vs.\ uninfected people will depend on the probability of infection given contact, as well as the effectiveness of the system for identifying suspicious contacts.

When a new infectious disease emerges, it is important to determine whether it will grow exponentially in a susceptible population, and if so at what rate $r$ \citep{Ma+14}.  The condition for positive exponential growth ($r>0$) is commonly expressed as $\Rnum>1$, where the basic reproduction number $\Rnum$ is the expected number of secondary infections arising from a typical infective individual in a completely susceptible population \citep{dietz1993estimation}.  Although the value of $\Rnum$ cannot completely characterize the dynamics of our model \citep{shaw2021what}, it does give a simple and widely accepted index for the difficulty of control, as well as an indication of the likely final size of an epidemic \citep{ma2006generality,miller2012note}.

In order to understand the effect of testing processes on epidemic dynamics, we expanded one of the simplest mechanistic epidemic models---the standard deterministic SIR model\citep{KermMcKe27,AndeMay91}---to include testing components. This model provides a sensible platform to link the modeling of epidemic and testing components and study their interaction. We studied the effects of testing intensity, rate of test return, and isolation efficacy, on transmission probability and epidemic dynamics when different levels of testing focus (from random to highly targeted) are in place.
